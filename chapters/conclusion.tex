% !TeX root = ../main.tex

\chapter{全文总结}

\section{工作总结}
在异地冗余的存储系统上执行分布式事务,是分布式事务中的最高峰。由于跨大陆网络链路的高延迟,这种情况下的事务处理的代价特别高,并且会带来额外的并发控制和数据复制开销。通常采用分层的协议(比如:Google Spanner)。这些分层协议包括OCC\cite{OCC}+FastPaxos\cite{FastPaxos},2PL+MultiPaxos\cite{Paxos}等等,他们都存在一个重大的缺陷,就是产生了很大的时延,需要多个RTT来完成协议内容。后来有一些工作在此基础上进行了改善,提供了一体化的设计思想,比如:Tapir(SOSP-15)、Janus(OSDI-16)、Ocean Vista(VLDB-19)、Slog (VLDB-19),这些工作在不同方面实现了较好的表现。本文研究了这些系统各自的特点,以及分析了这些系统在不同负载下的不足。主要工作包括:
\begin{itemize}
\item 调研分析不同系统的特点、归纳总结这些系统中存在的问题,以及为什么这样的设计在一定程度上能做到提升系统性能、优化系统设计的目标。
\item 在Janus的代码框架下,实现不同系统,在同一个代码框架下,可以使的系统之间的性能比较更加公平。
\item 优化TPC-C标准测试的实现,使其能够更好的支持不同的系统,减少不必要的跨数据中心协作。
\item 实现Retwis测试负载。
\item 实验比较并分析不同系统之间的特征和性能差异。
\end{itemize}

\section{未来工作展望}

异地冗余系统逐渐成为趋势,为了更好地实现系统冗余,部分冗余应该成为一个可能的方案。冗余全部数据到所有数据中心面临着巨大的存储开销,并且我们应该给予开发人员或设计者部分冗余的设计方案,这样会使得上层应用的开发更高效。而既有的系统在部分冗余的条件下,普遍存在大量的性能损失,因此设计一个能够满足部分冗余愿望的分布式事务协议是有价值的。

同时,基于Slog相同的切入点,分布式事务设计应该同时关注到事务发生的局部性特点,以此来优化系统设计。